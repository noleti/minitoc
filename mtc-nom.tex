%%
%% This is file `mtc-nom.tex',
%% generated with the docstrip utility.
%%
%% The original source files were:
%%
%% minitoc.dtx  (with options: `mtc-nom')
%% This is a generated file.
%% 
%% 
%% This file may be distributed and/or modified under the conditions of
%% the LaTeX Project Public License, either version 1.3 of this license
%% or (at your option) any later version.  The latest version of this
%% license is in:
%% 
%%    http://www.latex-project.org/lppl.txt
%% 
%% and version 1.3 or later is part of all distributions of LaTeX version
%% 2003/12/01 or later.
%% 
%% This work has the LPPL maintenance status "unmaintained".
%% 
%% This Current Maintainer of this work is <none>.
%% 
%% This work consists of all the files listed in the file `minitoc.l'
%% distributed with it.
%% Copyright 1993 1994 1995 1996 1997 1998 1999 2000
%%           2001 2002 2003 2004 2005 2006 2007 2008
%% Jean-Pierre F. Drucbert
%% <jean-pierre.drucbert@onera.fr>
\documentclass[oneside]{book}
\ProvidesFile{mtc-nom.tex}%
  [2007/04/02]
\usepackage[intoc]{nomencl}
\usepackage[tight]{minitoc}
\makenomenclature
\begin{document}
\dominitoc
\tableofcontents
\chapter{Angels}
\minitoc
\section{Main equations}
\begin{equation}
  a=\frac{N}{A}
\end{equation}%
\nomenclature{$a$}{The number of angels per unit area}%
\nomenclature{$N$}{The number of angels per needle point}%
\nomenclature{$A$}{The area of the needle point}%
The equation $\sigma = m a$%
\nomenclature{$\sigma$}{The total mass of angels per unit area}%
\nomenclature{$m$}{The mass of one angel}
follows easily.
\printnomenclature \mtcfixnomenclature
\chapter{Demons}
\minitoc
\section{False equations}
\begin{equation} i=\sqrt{-1} \end{equation}
\nomenclature{$i$}{The imaginary unit}%
\end{document}
%%%%%%%%%%%%%%%%%%%%%%%%%%%%%%%%%%%%%%%%%%%%%%%%%%%%%%%%%%%%%%%%%%%%%%%%%%%%%%%%%%%%%%%%%%%
%%
\endinput
%%
%% End of file `mtc-nom.tex'.
