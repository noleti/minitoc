%%
%% This is file `mtc-art.tex',
%% generated with the docstrip utility.
%%
%% The original source files were:
%%
%% minitoc.dtx  (with options: `mtc-art')
%% This is a generated file.
%% 
%% 
%% This file may be distributed and/or modified under the conditions of
%% the LaTeX Project Public License, either version 1.3 of this license
%% or (at your option) any later version.  The latest version of this
%% license is in:
%% 
%%    http://www.latex-project.org/lppl.txt
%% 
%% and version 1.3 or later is part of all distributions of LaTeX version
%% 2003/12/01 or later.
%% 
%% This work has the LPPL maintenance status "unmaintained".
%% 
%% This Current Maintainer of this work is <none>.
%% 
%% This work consists of all the files listed in the file `minitoc.l'
%% distributed with it.
%% Copyright 1993 1994 1995 1996 1997 1998 1999 2000
%%           2001 2002 2003 2004 2005 2006 2007 2008
%% Jean-Pierre F. Drucbert
%% <jean-pierre.drucbert@onera.fr>
%% mtc-art.tex
%% This file contains a set of tests for the minitoc.sty version #60
%% package. You can alter most of parameters to test.
%% article (\section must be defined)
\documentclass[12pt,a4paper]{article}
\ProvidesFile{mtc-art.tex}%
  [2007/06/06]
\usepackage{lipsum} % provides filling text
%% \usepackage{hyperref}      % If used, load it BEFORE minitoc
\usepackage[tight,insection]{minitoc}
\setcounter{secnumdepth}{5}   % depth of numbering of sectionning commands
\setcounter{tocdepth}{3}      % depth of table of contents
\setlength{\stcindent}{24pt}  % indentation of secttocs, default
%%                            % font for secttocs, default
\renewcommand{\stcfont}{\small\rmfamily\upshape\mdseries}%
%%                            % font for secttocs, subsections
%% \renewcommand{\stcSSfont}{\small\sf}%
%%                            % you can make experiments with
%%                            % \stcSSSfont, \stcPfont and \stcSPfont
%%                            % but it is ``fontomania''...
\raggedbottom                 % or \flushbottom, at your choice
%%% TEST: uncomment the next line to test
%%% resetting section number in each part
%%% \makeatletter \@addtoreset{section}{part} \makeatother
%%% END TEST
\begin{document}
\dosecttoc
\dosectlof[c]                   % center titles of the sectlofs
\dosectlot
\doparttoc                      % test of parttoc/partlof stuff
\dopartlof                      % added in version #15
\dopartlot                      % added in version #15
\faketableofcontents            % or \tableofcontents
\fakelistoffigures              % to check compatibility
\fakelistoftables               % to check compatibility
\part{First Part} \parttoc \partlof[r] \partlot
\twocolumn\sloppy               % the secttoc in twocolumn layout is ugly,
                                % but works. Ideas to make it better?
\section{AAAAA}                 % a section with a lot of sections
\secttoc[r]                     % secttoc title on the right
\mtcskip \sectlof %ADDED
\lipsum[1]
\subsection{S1} \lipsum[2]
\subsection{S2} \lipsum[3]
\subsection{S3} \lipsum[4]
\subsection*{S4}
%% \addcontentsline{toc}{starsubsection}{*S4*}
\lipsum[5]
\subsection{S5} \lipsum[6]
\subsection{S6} \lipsum[7]
\subsection{S7} \lipsum[8]
\subsection{S8} \lipsum[9]
\subsection{S9} \lipsum[10]
\subsection{S10} \lipsum[11]
\subsection{S11} \lipsum[12]
\subsection{S12} \lipsum[13]
\subsection{S13} \lipsum[14]
\subsection{S14} \lipsum[15]
\subsection{S15} \lipsum[16]
\subsection{S16} \lipsum[17]
\subsection{S17} \lipsum[18]
\subsection{S18} \lipsum[19]
\subsection{S19} \lipsum[20]
\subsection{S20} \lipsum[21]
\subsection{S21} \lipsum[22]
\subsection{S22} \lipsum[23]
\subsection{S23} \lipsum[24]
\subsection{S24} \lipsum[25]
\subsection{S25} \lipsum[26]
\subsection{S26} \lipsum[27]
\subsection{S27} \lipsum[28]
\subsection{S28} \lipsum[29]
\subsection{S29} \lipsum[30]
\subsection{S30} \lipsum[31]
\onecolumn\fussy         % back to one column
\section{BBBBB}
\secttoc
\mtcskip                 % put some skip here
\sectlof                 % a sectlof
\lipsum[32]
\subsection{T1} \lipsum[33]
\begin{figure}[t]        % tests compatibility with floating bodies
\setlength{\unitlength}{1mm}
\begin{picture}(100,50) \end{picture}
\caption{F1}             % (I have not tested tables, but it is similar)
\end{figure}
\FloatBarrier
\subsubsection[tt1]{TT1} % tests optional arg. of a sectionning command
\lipsum[34]
\paragraph{TTT1} \lipsum[35]
\subparagraph{TTTT1} \lipsum[36]
\begin{figure}[t]
\setlength{\unitlength}{1mm}
\begin{picture}(100,50) \end{picture}
\caption[f2]{F2}         % tests optional arg. of a caption
\end{figure}
\FloatBarrier
\subsection{T2} \lipsum[37]
\section*{CCCCC}         % tests a pseudo-section. should have no secttoc
%% \addstarredsection{CCCCC}
\mtcaddsection[CCCCC]
\secttoc \mtcskip \sectlof %ADDED
\lipsum[38]
\subsection{U1} \lipsum[39]
\subsubsection{UU1} \lipsum[40]
\paragraph{UUU1} \lipsum[41]
\subparagraph{UUUU1} \lipsum[42]
\subsection{U2} \lipsum[43]
\part{Second Part}
\parttoc
\partlof[c]
\partlot
%%                       % the following section should have no secttoc,
\section{DDDDD}          % but if you uncomment \secttoc,
%% \secttoc
\mtcskip \sectlof %ADDED
\lipsum[44]
\subsection{V1} \lipsum[45]
\subsubsection{VV1} \lipsum[46]
\paragraph{VVV1} \lipsum[47]
\subparagraph{VVVV1} \lipsum[48]
\begin{figure}[t]        % tests compatibility with floating bodies
\setlength{\unitlength}{1mm}
\begin{picture}(100,50) \end{picture}
\caption{F3}             % (I have not tested tables, but it is similar)
\end{figure}
\FloatBarrier
\lipsum[49] \subsection{V2} \lipsum[50]
\section{EEEEE}                 % this section should have a secttoc
{%                              % left brace, see below
\setcounter{secttocdepth}{3}    % depth of sect table of contents;
                                % try with different values.
\secttoc
\mtcskip \sectlof %ADDED
}                               % right brace
%% this pair of braces is used to keep local the change on secttocdepth.
\lipsum[51]
\subsection{W1}                 % with the given depth
\lipsum[52]
\subsubsection{WW1} \lipsum[53]
\paragraph{WWW1} \lipsum[54]
\begin{figure}[t]            % tests compatibility with floating bodies
\setlength{\unitlength}{1mm}
\begin{picture}(100,50) \end{picture}
\caption{F4}                 % (I have not tested tables, but it is similar)
\end{figure}
\FloatBarrier
bla bla bla bla bla bla bla bla bla bla bla
\subparagraph{WWWW1} \lipsum[55]
\subsection{W2} \lipsum[56]
\chapter*{}
\part{Appendices}
\parttoc \mtcskip
\partlof \mtcskip
\partlot
\FloatBarrier
\appendix
\section{Comments} \lipsum[57]
\secttoc
\mtcskip \sectlof %ADDED
\subsection{C1} \lipsum[58]
\subsection{C2} \lipsum[59]
\subsection{C3} \lipsum[60]
\begin{figure}[hb]        % tests compatibility with floating bodies
\setlength{\unitlength}{1mm}
\begin{picture}(100,50) \end{picture}
\caption{F5}              % (I have not tested tables, but it is similar)
\end{figure}
\FloatBarrier
\subsection{C4} \lipsum[61]
\FloatBarrier
\section{Evolution}
\secttoc
\sectlof % empty
\sectlot % empty
\lipsum[62]
\subsection{D1} \lipsum[63] \subsection{D2} \lipsum[64]
\subsection{D3} \lipsum[65] \subsection{D4} \lipsum[66]
\end{document}
%%%%%%%%%%%%%%%%%%%%%%%%%%%%%%%%%%%%%%%%%%%%%%%%%%%%%%%%%%%%%%%%%%%%%%%%%%%%%%%%%%%%%%%%%%%
%%
\endinput
%%
%% End of file `mtc-art.tex'.
