%%
%% This is file `mtc-mu.tex',
%% generated with the docstrip utility.
%%
%% The original source files were:
%%
%% minitoc.dtx  (with options: `mtc-mu')
%% This is a generated file.
%% 
%% 
%% This file may be distributed and/or modified under the conditions of
%% the LaTeX Project Public License, either version 1.3 of this license
%% or (at your option) any later version.  The latest version of this
%% license is in:
%% 
%%    http://www.latex-project.org/lppl.txt
%% 
%% and version 1.3 or later is part of all distributions of LaTeX version
%% 2003/12/01 or later.
%% 
%% This work has the LPPL maintenance status "unmaintained".
%% 
%% This Current Maintainer of this work is <none>.
%% 
%% This work consists of all the files listed in the file `minitoc.l'
%% distributed with it.
%% Copyright 1993 1994 1995 1996 1997 1998 1999 2000
%%           2001 2002 2003 2004 2005 2006 2007 2008
%% Jean-Pierre F. Drucbert
%% <jean-pierre.drucbert@onera.fr>
\documentclass[12pt]{report}
\ProvidesFile{mtc-mu.tex}%
  [2007/01/04]
\usepackage[tight]{minitoc}
\setlength{\mtcindent}{0pt}
\usepackage{wrapfig}
\newcommand{\LangSig}[1]{\textsc{[#1]}} % smallcaps
\begin{document}
\dominitoc \tableofcontents
\chapter{Mulspren}\label{chapter+mulspren}
\begin{wrapfigure}{r}{0.5\linewidth}
\begin{minipage}{\linewidth}
\vspace{-2.\baselineskip}
\minitoc
\vspace{-1.\baselineskip}
\end{minipage}
\end{wrapfigure}
The previous chapter examined many end-user programming environments
and found that most contain cognitive programming gulfs.
These gulfs were often created when programing environments used
multiple notations, and could manifest themselves in a variety of
usability problems, ranging from users being unable to understand
a program representation, to not wanting to execute their programs.
Conversely, the previous chapter also found circumstances where multiple
notations helped users understand programs.
It concluded that there was a place for multiple notation programming
environments, but developers had to be very careful to avoid creating
programming gulfs.
It concluded that there was a place for multiple notation programming
environments, but developers had to be very careful to avoid creating
programming gulfs.

This chapter introduces our programming environment, Mulspren.
Mulspren was designed to avoid these gulfs and gain the potential
benefits of multiple notations.
Users program using two notations, one similar to English and one
similar to conventional code.
Changes in one notation are immediately reflected in the other notation,
and users can move rapidly and seamlessly between the notations.
This is programming using dual notations.
When the program is executed, both notations are animated.
Mulspren's language signature is \LangSig{Re/Wr/Wa + Re/Wr/Wa + Wa}.

Papers describing Mulspren have been published in~\cite{Wright02-2}
and~\cite{Wright03-3}.

\section{section 1}
\section{section 2 bla bla bla bla bla bla bla bla bla bla bla
bla bla bla bla bla bla bla bla bla}
\section{section 3}
\section{section 4}
\section{section 5 bla bla bla bla bla bla bla bla bla bla bla
bla bla bla bla bla}
\begin{thebibliography}{1}
\bibitem{Wright02-2}
Tim Wright and Andy Cockburn.
\newblock Mulspren: a multiple language simulation programming
  environment.
\newblock In {\em HCC '02: Proceedings of the IEEE 2002 Symposia
  on Human Centric Computing Languages and Environments (HCC'02)},
  page 101, Washington, DC, USA, 2002. IEEE Computer Society.

\bibitem{Wright03-3}
Tim Wright and Andy Cockburn.
\newblock Evaluation of two textual programming notations for children.
\newblock In {\em AUIC '05: Proceedings of the Sixth Australasian
  conference on User interface}, pages 55--62, Darlinghurst, Australia,
  Australia, 2005.
  Australian Computer Society, Inc.
\end{thebibliography}
\end{document}
%%%%%%%%%%%%%%%%%%%%%%%%%%%%%%%%%%%%%%%%%%%%%%%%%%%%%%%%%%%%%%%%%%%%%%%%%%%%%%%%%%%%%%%%%%%
%%
\endinput
%%
%% End of file `mtc-mu.tex'.
